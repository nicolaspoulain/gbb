\documentclass[]{article}
\usepackage{amssymb,amsmath}
\usepackage{ifxetex,ifluatex}
\ifxetex
  \usepackage{fontspec,xltxtra,xunicode}
  \defaultfontfeatures{Mapping=tex-text,Scale=MatchLowercase}
  \newcommand{\euro}{€}
\else
  \ifluatex
    \usepackage{fontspec}
    \defaultfontfeatures{Mapping=tex-text,Scale=MatchLowercase}
    \newcommand{\euro}{€}
  \else
    \usepackage[utf8]{inputenc}
    \usepackage{eurosym}
  \fi
\fi
% Redefine labelwidth for lists; otherwise, the enumerate package will cause
% markers to extend beyond the left margin.
\makeatletter\AtBeginDocument{%
  \renewcommand{\@listi}
    {\setlength{\labelwidth}{4em}}
}\makeatother
\usepackage{enumerate}
\ifxetex
  \usepackage[setpagesize=false, % page size defined by xetex
              unicode=false, % unicode breaks when used with xetex
              xetex,
              colorlinks=true,
              linkcolor=blue]{hyperref}
\else
  \usepackage[unicode=true,
              colorlinks=true,
              linkcolor=blue]{hyperref}
\fi
\hypersetup{breaklinks=true, pdfborder={0 0 0}}
\setlength{\parindent}{0pt}
\setlength{\parskip}{6pt plus 2pt minus 1pt}
\setlength{\emergencystretch}{3em}  % prevent overfull lines
\usepackage[top=2cm,bottom=2cm,left=2cm,right=2cm,a4paper]{geometry}
\usepackage[frenchb]{babel}
%\usepackage{lscape}
%\usepackage{color}
%\definecolor{vert}{rgb}{0,0.5,0}
%\definecolor{bleu}{rgb}{0,0,0.5}
%\lstset{language=Python,basicstyle=\ttfamily\footnotesize,commentstyle=\color{vert},keywordstyle=\color{bleu}}

\title{Langages de balisage légers et logiciels de conversion de documents}
\author{Nicolas Poulain}

\begin{document}
\maketitle

\tableofcontents

\section{Présentation}

Pour saisir et mettre en forme des textes ou des documents textuels
comportant des insertions d'images, de figures ou de tableaux, on
utilise généralement un traitement de texte WYSIWYG\footnote{Un WYSIWYG
  pour \emph{What you see is what you get} est une interface utilisateur
  qui permet de composer visuellement le résultat voulu. C'est une
  interface intuitive : l'utilisateur voit directement à l'écran à quoi
  ressemblera le résultat final.}, propriétaire comme Microsoft Word ou
libre comme OpenOffice.

Les défauts majeurs de ces logiciels sont nombreux :

\begin{enumerate}[1.]
\item
  Le rédacteur d'un document se concentre presque autant autant sur le
  fond que sur la forme. Outre le temps passé, les conséquences sur le
  rendu sont nombreuses
  \begin{itemize}
  \item
    Les mises en forme les plus hétéroclites sont autorisées au dépens
    de la lisibilité ;
  \item
    Le résultat final est souvent discutable du point du vue de la
    typographie car les règles n'en sont pas respectées ni par
    l'utilisateur ni par le logiciel ;
  \item
    L'utilisation des styles est souvent anarchique et les documents mal
    structurés, ce qui rend la production automatique de sommaire ou
    d'index impossible ;
  \item
    L'insertion d'images ou de figures provoque des décalages mal
    maîtrisés.
  \end{itemize}
\item
  En ce qui concerne les documents longs, l'inclusion de documents
  annexes au sein du document maître donne des résultats aléatoires ;
\item
  L'interopérabilité n'est pas assurée entre les logiciels, elle ne
  l'est pas même entre les différentes versions d'un même logiciel, ce
  qui nous amène au dernier point ;
\item
  La pérénnité des documents n'est pas certaine puisque la compatibilité
  ascendante ne fonctionne pas toujours et qu'un document écrit il y a
  quelques années risque d'être perdu, faute du logiciel capable de le
  lire.
\end{enumerate}
À l'opposé de la composition dans un logiciel de traitement de texte, on
peut écrire des documents dans des langages de balisage. Il en existe de
nombreux : LaTeX, HTML, DocBook, etc. Les fichiers sont enregistrés au
format texte brut et doivent être interprétés par un logiciel afin
d'être consultés.

En ce qui concerne HTML et LaTeX, où pratiquement toutes les mises en
formes sont possibles, le problème vient de la difficulté à écrire les
balises\footnote{Dans le cas du format DocBook, c'est même humainement
  presque impossible de l'écrire à la main tant l'enchevêtrement des
  balises est inextriquable. On le génère avec un logiciel
  WYSIWYG\ldots{}}. Pour écrire un titre suivi d'une phrase contenant un
mot en gras puis une liste non numérotée, on saisira respecivement :

\begin{itemize}
\item
  en LaTeX

\begin{verbatim}
\section{Le titre du paragraphe}

Voici un mot en \textbf{gras} puis une liste :

\begin{enumerate}
 \item  c'est simple ;
 \item  c'est efficace.
\end{enumerate}
\end{verbatim}
\item
  en HTML

\begin{verbatim}
<h1>Le titre du paragraphe</h1>

<p>Voici un mot en <strong>gras</strong> puis une liste :</p>

<ul>
 <li> c'est simple ;</li>
 <li> c'est efficace.</li>
</ul>
\end{verbatim}
\end{itemize}
Comme on le voit, la syntaxe est accessible mais au goût de nombreux
utilisateurs il y a trop de commandes de mise en forme qui nuisent à la
lisibilité du texte lors de la saisie. C'est dommage car ces deux
formats ouverts et universels ont chacun leur avantage :

\begin{itemize}
\item
  HTML peut être lu sur n'importe quelle plateforme ou terminal du monde
  entier car ses spécifications, gérées le W3C\footnote{Un WYSIWYG pour
    \emph{What you see is what you get} est une interface utilisateur
    qui permet de composer visuellement le résultat voulu. C'est une
    interface intuitive : l'utilisateur voit directement à l'écran à
    quoi ressemblera le résultat final.}, sont respectées par les
  navigateurs web.
\item
  le logiciel LaTeX produit des documents de qualité unanimement
  reconnue. Il prend en charge la mise en page, l'utilisateur n'ayant
  qu'à se concentrer sur le fond et sa structure.
\end{itemize}
Il existe une alternative qui est à la fois simple, interopérable et
efficace : les langages de balisage légers.

Un langage de balisage léger est un langage utilisant une syntaxe
simple, conçue pour qu'un fichier en ce langage soit aisé à saisir avec
un éditeur de texte simple, et facile à lire dans sa forme non formatée.

Les wikis on grandement contribué à populariser ce type de langage. Le
principe est de saisir des balises accessibles aux non inités, un moteur
se chargeant de la conversion en HTML avant la publication.

\begin{verbatim}
Le titre du paragraphe
======================

Voici un mot en **gras** puis une liste :

* c'est simple ;
* c'est efficace.
\end{verbatim}
Avantages :

\begin{itemize}
\item
  les balises sont visuelles et le texte reste lisible ;
\item
  le nombre de balises et de règles à mémoriser est peu important ;
\item
  les balises étant constituées de cractères non alphabétiques, on peut
  utiliser un correcteur d'othographe.
\end{itemize}
C'est en 1995 que l'on trouva la solution de ce problème, avec la
création du premier langage Wiki, dont le but principal était de
permettre l'édition facile de pages web par tout un chacun, et dont
l'utilisateur actuel le plus célèbre est l'encyclopédie libre Wikipédia.
S'il y a presque autant de syntaxes différentes que de logiciels Wiki,
elles ont toutes la caractéristique d'utiliser des caractères textuels
simples et intuitifs pour donner les indications de formatage du texte.

Toujours le même exemple, une nouvelle section en MediaWiki :

= Nouvelle section Wiki =

et une en Setext :

\section{Nouvelle section Setext}

Mais pourquoi limiter ces langages de balisage léger à la seule
génération de HTML ? Pourquoi ne pas utiliser la même syntaxe pour
différentes cibles (appelées backends, targets ou writers selon les
logiciels), de manière à obtenir aussi bien une page web en HTML, qu'un
document en LaTeX pour l'impression, ou qu'une page de man pour un
logiciel ? Ce sont les logiciels qui poursuivent ce but qui
m'intéressent, ils constituent pour moi l'avenir de la bureautique
informatique, et j'ai été amené à les comparer pour en choisir un dans
lequel m'investir comme développeur.

Pandoc est un logiciel de conversion de documents permettant, à partir
d'un texte écrit dans un format très simple (Markdown), de faire une
page HTML, un document PDF ou encore un texte au format MediaWiki par
exemple.

L'auteur décrit son logiciel comme le couteau suisse de la création des
documents. Ce logiciel permet de convertir des textes simples dans de
nombreux formats, mais aussi de convertir différents formats entre eux,
avec une qualité en constante amélioration. Plus besoin de logiciels
lourds pour l'édition de vos documents : écrivez vos documents dans un
simple éditeur de texte, et convertissez ensuite votre fichier dans le
format de votre choix. Ce logiciel est idéal pour les documents
structurés avec tableaux et images.

\end{document}
